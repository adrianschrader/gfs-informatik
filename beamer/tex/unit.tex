%%
% Unit Tests
% Datum: 29.10.2016
% Autor: Adrian Schrader
%%
\section{Unit Tests}

\begin{frame}
  \frametitle{Definition}
  \begin{block}{Modul (Unit)}
    \begin{itemize}
      \item Funktionale, isolierbare Einheit
      \item Meist einzelne Klassen
    \end{itemize}
  \end{block}

  \begin{block}{Modultest (Unit Test)}
    \begin{itemize}
      \item sichert genau \emph{eine} sichtbare Eigenschaft
      \item bezieht sich auf \emph{eine} isolierte Komponente
      \item vollständig automatisierter Code (Testing Framework)
    \end{itemize}
  \end{block}
\end{frame}

\begin{frame}
  \frametitle{Merkmale}

  \begin{block}{Unit Tests sind\ldots}
    \begin{itemize}
      \item leicht verständlich
      \item voneinander unabhängig
      \item hochwertig und gewartet
      \item \glqq{}gemein\grqq{}
      \item schnell
    \end{itemize}
  \end{block}
\end{frame}

\begin{frame}
  \frametitle{Einsatz}

  \begin{columns}[T,onlytextwidth]
    \column{0.47\textwidth}
    \begin{exampleblock}{Vorteile}
      \begin{itemize}
        \item Zuverlässiger Code
        \item Überprüfbare Spezifikation
        \item Rückversicherung des Entwicklers
        \item Team-Management
      \end{itemize}
    \end{exampleblock}

    \column{0.47\textwidth}
    \begin{alertblock}{Nachteile}
      \begin{itemize}
        \item garantiert kein Zusammenspiel von Komponenten
        \item garantiert keine Performance
        \item \glqq doppelter\grqq\ Code
      \end{itemize}
    \end{alertblock}
  \end{columns}
\end{frame}

\begin{frame}[fragile]
  \frametitle{Implementierung}

  \begin{minted}{java}
import org.junit.Test;
import org.junit.Assert.*;

public class MatchBoxTest {
  public MatchBoxTest() {}

  @Test
  public void drawNotMoreThanThree() {
    MatchBox matchbox1 = new MatchBox(25);
    matchbox1.draw(4);
    assertEquals(22, matchbox1.count());
  }
}
  \end{minted}
\end{frame}
