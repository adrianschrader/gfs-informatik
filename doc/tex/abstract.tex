Die quantenmechanischen Entdeckungen in der ersten Hälte des 20. Jahrhunderts machen es durch präzise Spektroskopieverfahren möglich die Struktur von Molekülen, die für die Erklärung des Verhaltens und der Eigenschaften von Stoffen unersätzlich ist, mit physikalischen Prinzipien nachzuweisen. Durch die Eigenschaft des Kernspins gibt die NMR-Spektroskopie weitreichende Aufschlüsse ohne die Probe zu verändern oder gar zu zerstören und findet so auch in vielen weiteren Wissenschaftsbereichen wie etwa in der Medizin, der Forensik oder der Archäologie Anwendung. Dieses Paper bietet einen Einsteig in die Grundlagen dieses Verfahrens und zeigt anhand der \ce{^1H}-NMR-Spektroskopie welche Strukturaussagen möglich werden.

\emph{Gleichwertige Feststellung von Schülerleistungen} im \emph{Neigungsfach Chemie, Bolten}
